    %% ---------------------------------------------------------------------------------------------- %%
    % Os comandos abaixo ficam no início do documento, após o begin{document} na ordem que desejar
    %% ---------------------------------------------------------------------------------------------- %%

    % FICHA CATALOGRAFICA ==> Adicione com anexo
    % \begin{fichacatalografica}
    % \includepdf{fig_ficha_catalografica.pdf}
    % \end{fichacatalografica}

    % FOLHA DE APROVAÇÃO
    \orientadori{Prof. Dr. Beltrano de Tal}{Instituto Federal do Espírito Santo}{Orientador}
    \orientadorii{Profa. Dra. Beltrana de Tal}{Instituto Federal do Espírito Santo}{Coorientadora}
    \examinadori{Profa. Dra. Fulana de Tal}{Instituto Federal do Espírito Santo}{Examinadora}
    \examinadorii{Prof. Dr. Cicrano de Tal}{Instituto Federal do Espírito Santo}{Examinador}
    \approvaldate{01}{Dezembro}{2018}

    \begin{folhadeaprovacao}
    \begin{center}
        \centering
        
        {\bfseries\MakeUppercase\imprimirautor}
        \vspace*{1.5cm}
        
        {\bfseries\MakeUppercase\imprimirtitulo}
        
        \hspace{.40\textwidth}
        \begin{minipage}{9cm}
        \footnotesize
        \SingleSpacing
        \imprimirpreambulo
        \end{minipage}
        
        \vspace{1.0cm}  
        \imprimirapprovaldate
        \vspace{1.0cm}
        
        {\bfseries\MakeUppercase{comissão examinadora}}
        \vspace{-0.5cm}
        
        \assinaturaorientador
        \assinaturacoorientador
        \assinaturaexaminadori
        \assinaturaexaminadorii
        \assinaturaexaminadoriii
    \end{center}
    \end{folhadeaprovacao}    

    % DEDICATÓRIA
    \begin{dedicatoria}
    \vspace*{\fill}
    \noindent

    \begin{flushright}
        \textit{
        \textbf{Dedicatoria:} É um elemento opcional. Contém o oferecimento do trabalho adeterminada pessoa ou a pessoas (ASSOCIAÇÃO BRASILEIRA DE NORMAS TÉCNICAS, 2011b).} 
    \end{flushright}

    \vspace*{2.0cm}
    \end{dedicatoria}

    % AGRADECIMENTOS
    \begin{agradecimentos}
    É um elemento opcional. Localiza-se após a folha de aprovação e deve ser dirigido àqueles que realmente contribuíram, de maneira relevante,para a elaboração do trabalho. Deve-se utilizar uma linguagem simples (ASSOCIAÇÃO BRASILEIRA DE NORMAS TÉCNICAS, 2011b).
    \end{agradecimentos}

    % EPIGRAFE
    \begin{epigrafe}
    \vspace*{\fill}
    \begin{center}
        \textbf{Epigrafe:} É um elemento opcional. É uma citação relacionada ao assunto do trabalho desenvolvido, seguida da indicação de autoria (ASSOCIAÇÃO BRASILEIRA DE NORMAS TÉCNICAS, 2011b).   Deve-se seguir as regras do uso da citação NBR 10.520/2002. 
    \end{center}
    \end{epigrafe}

    % LISTA DE FIGURAS
    \pdfbookmark[0]{\listfigurename}{lof}
    \listoffigures*
    \cleardoublepage

    % LISTA DE TABELAS
    \pdfbookmark[0]{\listtablename}{lot}
    \listoftables*
    \cleardoublepage

    % LISTA DE QUADROS
    \pdfbookmark[0]{\listofquadrosname}{loq}
    \listofquadros*
    \cleardoublepage

    % LISTA DE ABREVIATUAS E SIGLAS
    \begin{siglas}
        \item[Ifes] Instituto Federal do Espírito Santo
    \end{siglas}

    % LISTA DE SÍBOLOS
    \begin{simbolos}
        \item[$\lambda$] Letra grega labda
    \end{simbolos}

    %% ---------------------------------------------------------------------------------------------- %%
    % Os comandos abaixo ficam ao final do documento, após o \postextual, na ordem que desejar
    %% ---------------------------------------------------------------------------------------------- %%

    % Apêndices
    \begin{apendicesenv}
        \input{apendices/ap1.tex}
    \end{apendicesenv}

    % Anexos
    \begin{anexosenv}
        \input{anexos/an1.tex}
    \end{anexosenv}

    % Índice Remissivo
    \phantompart
    \printindex