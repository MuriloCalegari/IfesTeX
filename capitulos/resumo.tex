% Para inserir no trabalho final, use: % Para inserir no trabalho final, use: % Para inserir no trabalho final, use: % Para inserir no trabalho final, use: \include{capitulos/resumo}

% RESUMO - PT
\begin{resumo}
  É a apresentação concisa e abreviada do conteúdo de um texto, do qual destacam-se as informações essenciais e os elementos de maior relevância. Em um resumo, o texto deve ser significativo, explicando o tema principal do documento, o objetivo, a metodologia, os resultados e as conclusões do trabalho. Resumir um texto consiste em fazer a exposição sucinta de um assunto, tendo em vista permitir ao leitor conhecer as informações mais importantes sobre este para, então,poder decidir sobre a conveniência de consultar ou não o texto integralmente (ASSOCIAÇÃO BRASILEIRA DE NORMAS TÉCNICAS, 2003a).

  \textbf{De 150 a 500 palavras – os resumos de trabalhos acadêmicos (teses, dissertações e outros) e relatórios técnico-científicos}

  \textbf{Palavras-chave}: Palavra-chave 1; Palavra-chave 2; ...; Palavra-chave N.
\end{resumo}

% RESUMO - EN
\begin{resumo}[Abstract]
\begin{otherlanguage*}{english}
  Enter the abstract here!
  
  \textbf{Keywords}: latex. abntex. text editoration.
\end{otherlanguage*}
\end{resumo}

% RESUMO - PT
\begin{resumo}
  É a apresentação concisa e abreviada do conteúdo de um texto, do qual destacam-se as informações essenciais e os elementos de maior relevância. Em um resumo, o texto deve ser significativo, explicando o tema principal do documento, o objetivo, a metodologia, os resultados e as conclusões do trabalho. Resumir um texto consiste em fazer a exposição sucinta de um assunto, tendo em vista permitir ao leitor conhecer as informações mais importantes sobre este para, então,poder decidir sobre a conveniência de consultar ou não o texto integralmente (ASSOCIAÇÃO BRASILEIRA DE NORMAS TÉCNICAS, 2003a).

  \textbf{De 150 a 500 palavras – os resumos de trabalhos acadêmicos (teses, dissertações e outros) e relatórios técnico-científicos}

  \textbf{Palavras-chave}: Palavra-chave 1; Palavra-chave 2; ...; Palavra-chave N.
\end{resumo}

% RESUMO - EN
\begin{resumo}[Abstract]
\begin{otherlanguage*}{english}
  Enter the abstract here!
  
  \textbf{Keywords}: latex. abntex. text editoration.
\end{otherlanguage*}
\end{resumo}

% RESUMO - PT
\begin{resumo}
  É a apresentação concisa e abreviada do conteúdo de um texto, do qual destacam-se as informações essenciais e os elementos de maior relevância. Em um resumo, o texto deve ser significativo, explicando o tema principal do documento, o objetivo, a metodologia, os resultados e as conclusões do trabalho. Resumir um texto consiste em fazer a exposição sucinta de um assunto, tendo em vista permitir ao leitor conhecer as informações mais importantes sobre este para, então,poder decidir sobre a conveniência de consultar ou não o texto integralmente (ASSOCIAÇÃO BRASILEIRA DE NORMAS TÉCNICAS, 2003a).

  \textbf{De 150 a 500 palavras – os resumos de trabalhos acadêmicos (teses, dissertações e outros) e relatórios técnico-científicos}

  \textbf{Palavras-chave}: Palavra-chave 1; Palavra-chave 2; ...; Palavra-chave N.
\end{resumo}

% RESUMO - EN
\begin{resumo}[Abstract]
\begin{otherlanguage*}{english}
  Enter the abstract here!
  
  \textbf{Keywords}: latex. abntex. text editoration.
\end{otherlanguage*}
\end{resumo}

% RESUMO - PT
\begin{resumo}
  É a apresentação concisa e abreviada do conteúdo de um texto, do qual destacam-se as informações essenciais e os elementos de maior relevância. Em um resumo, o texto deve ser significativo, explicando o tema principal do documento, o objetivo, a metodologia, os resultados e as conclusões do trabalho. Resumir um texto consiste em fazer a exposição sucinta de um assunto, tendo em vista permitir ao leitor conhecer as informações mais importantes sobre este para, então,poder decidir sobre a conveniência de consultar ou não o texto integralmente (ASSOCIAÇÃO BRASILEIRA DE NORMAS TÉCNICAS, 2003a).

  \textbf{De 150 a 500 palavras – os resumos de trabalhos acadêmicos (teses, dissertações e outros) e relatórios técnico-científicos}

  \textbf{Palavras-chave}: Palavra-chave 1; Palavra-chave 2; ...; Palavra-chave N.
\end{resumo}

% RESUMO - EN
\begin{resumo}[Abstract]
\begin{otherlanguage*}{english}
  Enter the abstract here!
  
  \textbf{Keywords}: latex. abntex. text editoration.
\end{otherlanguage*}
\end{resumo}