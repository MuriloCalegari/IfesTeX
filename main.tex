\documentclass[
  12pt,             % tamanho da fonte
  openany,			% capítulos começam em qualquer número de página
  oneside,          % Imprime somente no anverso da folha
  a4paper,          % tamanho do papel. 
  chapter=TITLE,    % títulos de capítulos convertidos em letras maiúsculas
  section=TITLE,    % títulos de seções convertidos em letras maiúsculas
  english,          % idioma adicional para hifenização
  french,           % idioma adicional para hifenização
  spanish,          % idioma adicional para hifenização
  brazil,           % o último idioma é o principal do documento
  hyphens,          % Também quebra URLs em hífens
  arial             % pode ser arial ou times
]{iftex}

%% Arial: 
% Para usar times-new-roman, comente as três linhas abaixo
%\usepackage{helvet}
%\renewcommand{\familydefault}{\sfdefault}
%\renewcommand{\ABNTEXchapterfont}{\bfseries}

%% Macros de dados do documento
\autor{Fulano de Tal \\ Sicrano Beltrano da Silva}
\titulo{Um título interessante:\\Subtítulo do trabalho}
\instituicao{Instituto federal do espírito santo}
\curso{Curso técnico em LaTeX integrado ao GitHub}
\local{Cidade maravilhosa}
\data{2018}

\tipotrabalho{Trabalho de Graduação}
\preambulo{Trabalho apresentado à disciplina de LaTeX do Curso Técnico em LaTeX do Instituto Federal do Espirito Santo, como requisito parcial para avaliação.}
\orientador{Zé da Silva}{Orientador}

\begin{document}
  % Seleciona o idioma do documento
  \selectlanguage{brazil}
  
  % Insere a pasta onde estão contidas as figuras
  \graphicspath{{figuras/}}

  % --------------------------------------------------- %
  % ELEMENTOS PRE-TEXTUAIS %
  % --------------------------------------------------- %

  \pretextual

  % CAPA E CONTRA-CAPA
	\imprimircapa
	\imprimirfolhaderosto
    \clearpage
  
  % LISTA DE FIGURAS
  \pdfbookmark[0]{\listfigurename}{lof}
  \listoffigures*
  \cleardoublepage

  % LISTA DE TABELAS
  \pdfbookmark[0]{\listtablename}{lot}
  \listoftables*
  \cleardoublepage
 
  % SUMÁRIO
  \pdfbookmark[0]{\contentsname}{toc}
  \tableofcontents*
  \cleardoublepage

  % --------------------------------------------------- %
  % ELEMENTOS TEXTUAIS %
  % --------------------------------------------------- %
    
  \textual
  \pagestyle{simple}
  \aliaspagestyle{chapter}{simple}
  
  \input{capitulos/introducao}

  \chapter[Conclusão]{Conclusão}
\lipsum[2]
\begin{table}[h]
    \IBGEtab{%
        \caption{Um Exemplo de tabela alinhada que pode ser longa ou curta,
        conforme padrão IBGE.}%
        \label{tabela-ibge}
    }{%
        \begin{tabular}{ccc}
            \toprule
            Nome & Nascimento & Documento \\
            \midrule
            Teste & 11/11/1111 & 111.111.111-11 \\
            \bottomrule
        \end{tabular}%
    }{%
        \fonte{Produzido pelos autores}%
    }
    \vspace{-0,6cm}
\end{table}
\lipsum[3]
    
  % --------------------------------------------------- %
  % ELEMENTOS POS-TEXTUAIS %
  % --------------------------------------------------- %

  \postextual
  
  % Referencias Bibliográficas
  
  % Citações ocultas para aparecer nas referências
  
  \citeoption{minhasopcoes}
  \bibliographystyle{abntex2-alf} % Autor-Data
  \bibliography{bibliografia}
  
\end{document}